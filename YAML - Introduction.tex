%!TEX TS-program = xelatex

\documentclass{beamer}

% ============
% = Packages =
% ============

\usepackage{grffile}
\usepackage{minted}
\usepackage{polyglossia}
\usepackage{tikz}
\usepackage{xltxtra}
\usepackage{xelatexemoji}

% ==============
% = Attributes =
% ==============

\newcommand{\Title}{YAML}
\newcommand{\TitleDescription}{Einführung}
\newcommand{\Subject}{Einführende Präsentation zur\\
                      Daten-Serialisierungssprache YAML}
\newcommand{\KeyWords}{YAML, Serialisierung}
\newcommand{\LeftFooter}{\Title~—~\TitleDescription}

\newcommand{\Author}{René Schwaiger}
\newcommand{\Mail}{\href{mailto:sanssecours@me.com}{sanssecours@me.com}}

% ============
% = Settings =
% ============

% Code
\usemintedstyle{rainbow_dash}

% Colors
\definecolor{Aqua}{rgb}{0, 0.56, 1}

% Document Properties
\hypersetup{pdftitle=\Title, pdfsubject=\Subject, pdfauthor=\Author,
            pdfkeywords = \KeyWords}

\title{\hspace{-3mm}\textbf{\textcolor{black}{Y}
       \hspace{-3.6mm}\textcolor{red}{A}\\[-10pt]
       \hspace{+3mm}\textcolor{black}{ML}}}
\date{\textit{\Subject}}

% Fonts
\setmainfont{Avenir}
\setsansfont{Ubuntu}
\setmonofont{Ubuntu Mono}
\newfontfamily\Helvetica{Helvetica}

% Language
\setmainlanguage{english}
\setotherlanguage{german}

% Theme
\usetheme[headheight=10pt, footheight=10pt]{boxes}
\addheadbox{header}{~~ \Title ~~~~~~
                    \insertframenumber{} / \inserttotalframenumber}
\addfootbox{footer}{~~ \insertsection ~ \insertsubsection}
\beamertemplatenavigationsymbolsempty
\setbeamertemplate{itemize item}[circle]

\setbeamercolor{alerted text}{fg=orange}
\setbeamercolor{frametitle}{fg=Aqua}
\setbeamercolor{block title}{fg=black}
\setbeamercolor{itemize item}{fg=black}
\setbeamercolor{structure}{fg=black}
\setbeamercolor{title}{fg=Aqua}
\setbeamercolor{titlelike}{fg=Aqua}
\setbeamercolor{left footer}{fg=white, bg=black}
\setbeamercolor{right footer}{fg=white, bg=black}
\setbeamercolor{footer}{fg=white, bg=black}
\setbeamercolor{header}{fg=white, bg=black}
\setbeamercovered{transparent=20}

\setbeamerfont{title}{series=\Helvetica, size={\fontsize{48}{48}}}

% ============
% = Document =
% ============

\begin{document}

{
  \usebackgroundtemplate{
    \tikz\node[opacity=0.1]{
      \parbox[c][\paperheight][c]{\paperwidth}{
        \centering
        \includegraphics[width=0.85\textwidth]{Figures/Elektra - Logo}
      }
    };
  }
  \begin{frame}[plain]
      \titlepage
  \end{frame}
}

\begin{frame}{Inhalt}
    \tableofcontents
\end{frame}

\section{🐪 Allgemeines}

\begin{frame}[fragile]{🐪 Allgemeines}
  \begin{minted}[autogobble]{yaml}
    - Abkürzung:
       - Früher: Yet Another Markup Language
       - Heute: YAML Ain’t Markup Language
    - Aktuelle Version: YAML 1.2 (2009)
    - Superset von JSON # seit YAML 1.2
    - 3 verschiedene Datentypen:
      - Skalar:
        - "Hello World"
        - '👋 🌍'
        - 123
      - Sequenz: [Text, 'Text', "Text", 123.5]
      - Map: {🔑: Value, see-no-evil monkey: 🙈}
    - Hauptziel: “YAML is easily readable by humans.”
    # Kompletter Text auf dieser Folie ist valides YAML
  \end{minted}
\end{frame}

\subsection{😱 Flow Style}

\begin{frame}[fragile]{😱 Flow Style}
  \begin{block}{Dokument der vorhergehenden Folie in reinem Flow-Style}
    \begin{minted}[autogobble]{yaml}
    [{'Abkürzung':
       [{'Früher': 'Yet Another Markup Language'},
        {'Heute': 'YAML Ain’t Markup Language'}]},
     {'Aktuelle Version': 'YAML 1.2 (2009)'},
     'Superset von JSON',
     {'3 verschiedene Datentypen':
        [{'Skalar': ['Hello World', '👋 🌍', 123]},
         {'Sequenz': ['Text', 'Text', 'Text', 123.5]},
         {'Map': {'🔑': 'Value', 'see-no-evil monkey': '🙈'}}]},
     {'Hauptziel': '“YAML is easily readable by humans.”'}]
    # Kompletter Text auf dieser Folie ist valides YAML
    \end{minted}
  \end{block}
\end{frame}

\end{document}
