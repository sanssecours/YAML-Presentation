%!TEX TS-program = xelatex

\documentclass{beamer}

% ============
% = Packages =
% ============

\usepackage{etoolbox}
\usepackage{minted}
\usepackage{polyglossia}
\usepackage{tcolorbox}
\usepackage{tikz}
\usepackage{xltxtra}
\usepackage{xelatexemoji}
\usepackage{unicode-math}

% ==============
% = Attributes =
% ==============

\newcommand{\Title}{YAML}
\newcommand{\Subject}{Introductory Presentation about\texorpdfstring{\\}{}
                      the Data Serialization Language YAML}
\newcommand{\KeyWords}{YAML, Serialization}

\newcommand{\Author}{René Schwaiger}
\newcommand{\Mail}{\href{mailto:sanssecours@me.com}{sanssecours@me.com}}

% ============
% = Settings =
% ============

% Boxes
\newtcbox{\codebox}{
  on line,
  arc = 2pt,
  colback = gray!10!white,
  boxsep = 2pt,
  left = 1pt,
  right = 1pt,
  top = 1pt,
  bottom = 1pt,
  boxrule = 0pt,
  bottomrule = 0pt,
  toprule = 0pt
}

\newtcolorbox{code-boxed}{
  on line,
  arc = 2pt,
  colback = gray!10!white,
  boxsep = 2pt,
  left = 1pt,
  right = 1pt,
  top = 1pt,
  bottom = 1pt,
  boxrule = 0pt,
  bottomrule = 0pt,
  toprule = 0pt
}

% Code
\usemintedstyle{rainbow_dash}
\BeforeBeginEnvironment{minted}{\begin{code-boxed}}
\AfterEndEnvironment{minted}{\end{code-boxed}}

% Colors
\definecolor{Aqua}{rgb}{0, 0.56, 1}

% Document Properties
\hypersetup{pdftitle=\Title, pdfsubject=\Subject, pdfauthor=\Author,
            pdfkeywords=\KeyWords}

\title{\texorpdfstring{\vspace{2cm}\\
       \hspace{-3mm}\textbf{\textcolor{black}{Y}
       \hspace{-3.6mm}\textcolor{red}{A}\\[-10pt]
       \hspace{+3mm}\textcolor{black}{ML}}\vspace{-1cm}}{YAML}}
\date{\textit{\Subject}}

% Fonts
\setmainfont{Avenir}
\setsansfont{Ubuntu}
\setmonofont{Ubuntu Mono}
\newfontfamily\Helvetica{Helvetica}

% Language
\setmainlanguage{english}
\setotherlanguage{german}

% Theme
\usetheme[headheight = 10pt, footheight = 10pt]{boxes}
\addheadbox{header}{~~ \Title \hfill
                    \insertframenumber{} / \inserttotalframenumber ~~}
\addfootbox{footer}{~~ \insertsection ~ \insertsubsection ~ \insertsubsubsection}
\beamertemplatenavigationsymbolsempty
\setbeamertemplate{itemize item}[circle]

\setbeamercolor{alerted text}{fg = orange}
\setbeamercolor{frametitle}{fg = Aqua}
\setbeamercolor{block title}{fg = black}
\setbeamercolor{itemize item}{fg = black}
\setbeamercolor{structure}{fg = black}
\setbeamercolor{title}{fg = Aqua}
\setbeamercolor{titlelike}{fg = Aqua}
\setbeamercolor{left footer}{fg = white, bg = black}
\setbeamercolor{right footer}{fg = white, bg = black}
\setbeamercolor{footer}{fg = white, bg = black}
\setbeamercolor{header}{fg = white, bg = black}
\setbeamercovered{transparent = 20}

\setbeamerfont{title}{series=\Helvetica, size={\fontsize{48}{48}}}

\AtBeginSection
{
  \begin{frame}{}
    \tableofcontents[
      sectionstyle = show/shaded,
      subsectionstyle = show/shaded/hide,
      subsubsectionstyle = show/hide,
  ]
  \end{frame}
}

\AtBeginSubsection
{
  \begin{frame}{}
    \tableofcontents[
      sectionstyle = shaded/hide,
      subsectionstyle = show/shaded/hide,
      subsubsectionstyle = show/show/hide,
  ]
  \end{frame}
}

% ==========
% = Macros =
% ==========

\newcommand{\code}[1]{
  \codebox{\mintinline{yaml}|#1|}
}

% ============
% = Document =
% ============

\begin{document}

{
  \usebackgroundtemplate{
    \tikz\node[opacity = 0.1]{
      \parbox[c][\paperheight][c]{\paperwidth}{
        \centering
        \includegraphics[width = 0.85\textwidth]{Figures/Logo}
      }
    };
  }
  \begin{frame}[plain]
      \titlepage
  \end{frame}
}

\section{👁 Overview}
\subsection{🐪 Introduction}

\begin{frame}[fragile]{🐪 Introduction}
  \begin{minted}[autogobble]{yaml}
    - Abbreviation:
       - At First: Yet Another Markup Language
       - Today: YAML Ain’t Markup Language
    - Current Version: YAML 1.2 (2009)
    - Superset of JSON # since YAML 1.2
    - Two Styles: Flow Style (Indicator Based) and
                  Block Style (Indentation Based)
    - 3 Different Data Types:
      - Scalar:
        - "Hello World"
        - '👋 🌍'
        - 123
      - Sequence: [Text, 'Text', "Text", 123.5]
      - Map: {🔑: Value, see-no-evil monkey: 🙈}
    - Design Goal: “YAML is easily readable by humans.”
    # All of the text on this slide is valid YAML
  \end{minted}
\end{frame}

\subsection{😱 Flow Style}

\begin{frame}[fragile]{😱 Flow Style}
  \begin{block}{Previous Document in Pure Flow Style}~\\
    \begin{minted}[autogobble]{yaml}
      [{'Abbreviation':
        [{'At First': 'Yet Another Markup Language'},
         {'Today': 'YAML Ain’t Markup Language'}]},
       {'Current Version': 'YAML 1.2 (2009)'},
       'Superset of JSON',
       {'Two Styles': 'Flow Style (Indicator Based) and
                       Block Style (Indentation Based)'},
       {'3 Different Data Types':
         [{'Scalar': ['Hello World', '👋 🌍', 123]},
          {'Sequence': ['Text', 'Text', 'Text', 123.5]},
          {'Map': {'see-no-evil monkey':
           '🙈', '🔑': 'Value'}}]},
       {'Design Goal':
        '“YAML is easily readable by humans.”'}]
     \end{minted}
  \end{block}
\end{frame}

\section{🐣 Basic Data Types}

\subsection{📏  Scalar}

\begin{frame}[fragile]{📏 Scalar}
  \begin{itemize}
    \item Most Basic Type
    \item Saves Numbers, Strings
    \item Examples (Flow Style)
    \begin{itemize}
      \item Strings
      \begin{itemize}
        \item \codebox{\mintinline{yaml}|Plain String|}
        \item \codebox{\mintinline{yaml}|'Single Quoted String'|}
        \item \codebox{\mintinline{yaml}|"Double Quoted String"|}
        \item \codebox{\mintinline{yaml}|'1337'|}
        \item \codebox{\mintinline{yaml}|!!str 1337|} (Explicit Type Specification)
      \end{itemize}
      \item Integers
      \begin{itemize}
        \item \codebox{\mintinline{yaml}|1337|}
        \item \codebox{\mintinline{yaml}|!!int 1337|}
      \end{itemize}
      \item Floating Point Numbers
      \begin{itemize}
        \item \codebox{\mintinline{yaml}|0.1337|}
        \item \codebox{\mintinline{yaml}{!!float 0.1337}}
      \end{itemize}
    \end{itemize}
  \end{itemize}
\end{frame}

\subsection{📋 Sequence}

\begin{frame}[fragile]{📋 Sequence}
  \begin{block}{Sequence of Values \textcolor{gray}{(Array, List)}}
    \vspace{0.3cm}

    \begin{minted}[autogobble]{yaml}
      # Flow Style
      [🍎, 🍊, [Sugar, Eggs, Chocolate]]
    \end{minted}

    \vspace{0.1cm}

    \begin{minted}[autogobble]{yaml}
      # Block Style
      - 🍎
      - 🍊
      - - Sugar
        - Eggs
        - Chocolate
    \end{minted}

    \vspace{0.1cm}

    \begin{minted}[autogobble]{yaml}
      # Mixed
      - 🍎
      - 🍊
      - [Sugar, Eggs, Chocolate]
    \end{minted}
  \end{block}
\end{frame}

\subsection{🗺 Map}

\begin{frame}[fragile]{🗺 Map}
  \begin{block}{Unordered Pairs of Values}
    \textcolor{gray}{(Associative Array, Dictionary, Hash, Hashmap, Hashtable)}
    \vspace{0.2cm}

    \begin{minted}[autogobble]{yaml}
      # Flow-Style
      { Austria: Vienna,
        South Africa: {Executive: Pretoria,
          Judicial: Bloemfontein, Legislative: Cape Town}}
    \end{minted}

    \vspace{0.1cm}

    \begin{minted}[autogobble]{yaml}
      # Block-Style
      Austria: Vienna
      South Africa:
        Executive:   Pretoria
        Judicial:    Bloemfontein
        Legislative: Cape Town
    \end{minted}

    \vspace{0.1cm}

    \begin{minted}[autogobble]{yaml}
      # Complex Key (`?` optional für simple Keys)
      ? [Pretoria, Bloemfontein, Cape Town]
      : South Africa
    \end{minted}
  \end{block}
\end{frame}

\section{✨ Features}
\subsection{📝 Scalar Input}

\subsubsection{🌊 Flow Style}

\begin{frame}[fragile]{🌊 Flow Style}
  \begin{enumerate}
    \item \code{Plain}
    \begin{itemize}
      \item Most Readable Style
      \item Combinations \codebox{\texttt{\#~~}} , \code{: ⁣} not possible
      \item First Character: Indicators (\code{:}, \code{[}, …) only under certain circumstances
    \end{itemize}
    \item \code{'Single Quoted'}
    \begin{itemize}
      \item Provides good readability
      \item Apart from \code{'} no special characters
    \end{itemize}
    \item \code{"Double Quoted"}
    \begin{itemize}
      \item Supports Escape Sequences:\\ \code{\n}, \codebox{\texttt{{\textbackslash}x0a}}, \codebox{\texttt{{\textbackslash}u000a}}, \codebox{\texttt{{\textbackslash}U0000000a}}, …
      \item Only style that supports arbitrary characters
    \end{itemize}
  \end{enumerate}
\end{frame}

\subsubsection{🗒 Block Style}

\begin{frame}[fragile, allowframebreaks]{🗒 Block Style}

  \begin{itemize}
    \item Supports long blocks of data
    \item No special characters
  \end{itemize}

  \begin{enumerate}
    \item
      \begin{minted}[autogobble]{yaml}
        |
          Literal
          Style
        # "Literal\nStyle"
      \end{minted}
      \begin{itemize}
        \item No changes to newline characters (\code{\n})
      \end{itemize}

    \item
      \begin{minted}[autogobble]{yaml}
        >
          Folded
          Style
        # "Folded Style"
      \end{minted}
      \begin{itemize}
        \item Replaces first newline character with space, if the next non-empty line does not contain leading whitespace characters
    \end{itemize}

  \end{enumerate}

  \newpage
  \hspace{0pt}
  \vspace{-1cm}
  \vfill
  \begin{block}{Comparison}\
    \begin{columns}

      \column{.5\textwidth}
      Literal:\\[0.2cm]
      \begin{minted}[autogobble, showspaces, spacecolor = lightgray, space=·]{yaml}
        |
          One
          Two

          List:
           * Item 1
           * Item 2

        # "One\nTwo\n\nList:\n\ \
        #  * Item 1\n * Item 2\n"
      \end{minted}

      \column{.5\textwidth}
      Folded:\\[0.2cm]
      \begin{minted}[autogobble, showspaces, spacecolor = lightgray, space=·]{yaml}
        >
          One
          Two

          List:
           * Item 1
           * Item 2

        # "One Two\nList:\n\ \
        #  * Item 1\n * Item 2\n"
      \end{minted}

    \end{columns}
  \end{block}
  \hspace{0pt}
  \vfill

  \newpage
  \begin{block}{Block Scalar Header}
    \begin{enumerate}
      \item Indentation Header \codebox{\texttt{|4}}, \code{>2}, \code{>5}, …
      \begin{itemize}
        \item Specifies indentation of block
        \item No indentation header $⇔$ Automatic detection
        \item First lines contain additional whitespace $⇔$ \\
              \alert{Auto-Detection fails}
      \end{itemize}
      \item Chomping Header \codebox{\texttt{|-}}, \codebox{\texttt{|+}}, \code{>-}, \code{>+}
      \begin{itemize}
        \item Specifies removal of empty lines \alert{at the end}
        \begin{enumerate}
          \item Strip \code{-}: Remove trailing whitespace
          \item Keep \code{+}: Save trailing whitespace
          \item Clip (No Indicator): Replace empty lines with \code{\n}
        \end{enumerate}
      \end{itemize}
    \end{enumerate}
    Arbitrary combinations (e.g. \codebox{\texttt{|-8}}, \code{>4+}) possible
  \end{block}

\end{frame}

\subsection{🗃 Stream}

\begin{frame}[fragile]{🗃 Stream}
  \begin{itemize}
    \item YAML Stream: Sequence of Multiple Documents
    \item Document End Marker \code{...} separates documents
    \item Optional Directives End Marker \code{---} separates \alert{directives} from content
  \end{itemize}
  \begin{minted}[autogobble]{yaml}
    %YAML 1.2 # Directive
    ---
    "Hello First Document"
    ...
    Document without Directives
    ...
    %TAG ! example.uri
    ---
    'Third Document'
  \end{minted}
\end{frame}

\subsection{📢 Directives}

\begin{frame}[fragile]{📢 Directives}
  Directives before document content specify:
  \begin{enumerate}
    \item YAML Version \codebox{\texttt{\%YAML}}
    \item Prefix for tag shorthands \codebox{\texttt{\%TAG}}
    \begin{itemize}
      \item Predefined primary and secondary tag shorthand:
      \begin{minted}[autogobble]{yaml}
        %TAG !  !
        %TAG !! tag:yaml.org,2002:
      \end{minted}
    \end{itemize}
  \end{enumerate}
  \begin{minted}[autogobble]{yaml}
    # Directives
    %YAML 1.2                       # YAML Version
    %TAG ! tag:yaml.org,2002:       # Primary Tag Handle
    %TAG !! tag:yaml.org,2002:      # Secondary Tag Handle
    %TAG !named! tag:yaml.org,2002: # Named Tag Handle
    --- # Directives End Marker
    [!!str 6, !str 6, !named!str 6, ! 6,
     '6', !<tag:yaml.org,2002:str> 6] # 6 times "6"
  \end{minted}
\end{frame}

\subsection{🐠🐙🐶 Types}

\subsubsection{🏷 Tags}

\begin{frame}[allowframebreaks, fragile]{🏷 Tags}
  \begin{itemize}
    \item Optional \alert{Tag} specifies data type
    \item Without tag $⇒$ automatic recognition of type
    \item 3 Different Types of Tags
    \begin{enumerate}
      \item Verbatim \code{!<!local>}, \code{!<global>}
      \item Shorthand \code{!suffix}, \code{!!suffix}, \code{!named!suffix}
      \item Non-Specific \code{!}
    \end{enumerate}
    \item Verbatim Tags
    \begin{description}
      \item [Local Tag] Different semantics in different context (e.g. \code{!<!ruby/object:Set>})
      \item [Global Tag] Semantics are always the same\\
            (e.g. \code{!<tag:yaml.org,2002:int>})
    \end{description}
    \item Non Specific Tag \code{!} (Only for non-plain scalars)
    \item YAML resolves shorthand tags and non-specific tags to verbatim tags
  \end{itemize}

  \newpage

  \begin{block}{Shorthand/Non-Specific Tag $⇒$ Verbatim Tag}~\\
    \begin{minted}[autogobble]{yaml}
      # %TAG  shorthand     definition
      %TAG     !name!    tag:yaml.org,2002:
      ---
      # shorthand|suffix ➡️ !<definition|suffix>
      !suffix      value # !<!suffix>                  value
      !!suffix     value # !<tag:yaml.org,2002:suffix> value
      !name!suffix value # !<tag:yaml.org,2002:suffix> value
      !            value # !<tag:yaml.org,2002:str>    value
      ...
      %TAG !name! lego.block/ # New prefix for shorthand
      ---
      !name!suffix value # !<lego.block/suffix> value
    \end{minted}
  \end{block}

\end{frame}

\subsubsection{📚 Schemas}

\begin{frame}[allowframebreaks, fragile]{📚 Schemas}
  \begin{itemize}
    \item 3 Predefines Schema (Categories) for Types (Tags)
    \begin{enumerate}
      \item Failsafe Schema
      \begin{description}
        \item [String] \code{!!str Text}
        \item [Sequence] \code{!!seq [Item1, Item2]}
        \item [Mapping] \code{!!map {Key: Value}}
      \end{description}
      \item JSON Schema (Failsafe Schema + JSON Types)
      \begin{description}
        \item [Null] \code{!!null null: value of null key}
        \item [Boolean] \code{!!bool false}
        \item [Integer] \code{!!int -42}
        \item [Float] \code{!!float .nan}
      \end{description}
      \item Core Schema (JSON Schema + Additional Input Possibilities)
      \begin{itemize}
        \item Octal- \code{0o123} and Hex-Format \code{0xfefe}
        \item Multiple Notations for same value \\
              e.g. \code{null}, \code{Null}, \code{~}, \code{NULL}
      \end{itemize}
    \end{enumerate}
  \end{itemize}

  \newpage
  \begin{itemize}
    \item Additional Types (\textcolor{orange}{\url{http://yaml.org/type}})
    \begin{description}[labelwidth = 1cm]
      \item [Ordered Map] \code{!!omap [first: 1, second: 2]}
      \item [Set] \code{!!set {No, Order}}
      \item [Binary] \code{!!binary "YmFzZTY0Cg=="}
      \item [Time] \code{!!timestamp 1337-12-24}
      \item […]
    \end{description}
    \item Definition of custom types via local tags\\[0.5cm]
    \begin{minted}[autogobble]{yaml}
      # print(Set.new([false, 1, '2']).to_yaml)
      --- !ruby/object:Set
      hash:
        false: true
        1: true
        '2': true
    \end{minted}
  \end{itemize}
\end{frame}

\subsection{⚓️ Anchors \& Aliases}

\begin{frame}[fragile]{⚓️ Anchors \& Aliases}
  \begin{itemize}
    \item Use same data multiple times
    \item Anchor \code{&name} specifies data location
    \item Reference \code{*name} inserts data at current location~\\[0.5cm]
    \begin{minted}[autogobble]{yaml}
      green: &green
        🌳🌱🌿🌱🌿🌱🌿🌳
      flowers: &flowers
        🌳🌸🌼🌻🌺🌼🌸🌳
      garden:              # puts(data['garden']):
        - *green           # 🌳🌱🌿🌱🌿🌱🌿🌳
        - *flowers         # 🌳🌸🌼🌻🌺🌼🌸🌳
        - 🌳🌱🐞🌱🌱🐛🌱🌳 # 🌳🌱🐞🌱🌱🐛🌱🌳
        - *flowers         # 🌳🌸🌼🌻🌺🌼🌸🌳
        - *green           # 🌳🌱🌿🌱🌿🌱🌿🌳
    \end{minted}
  \end{itemize}
\end{frame}

\end{document}
