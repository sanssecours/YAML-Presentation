%!TEX TS-program = xelatex

\documentclass{beamer}

% ============
% = Packages =
% ============

\usepackage{etoolbox}
\usepackage{minted}
\usepackage{polyglossia}
\usepackage{tcolorbox}
\usepackage{tikz}
\usepackage{xltxtra}
\usepackage{xelatexemoji}
\usepackage{unicode-math}

% ==============
% = Attributes =
% ==============

\newcommand{\Title}{YAML}
\newcommand{\Subject}{Einführende Präsentation zur\texorpdfstring{\\}{}
                      Daten-Serialisierungssprache YAML}
\newcommand{\KeyWords}{YAML, Serialisierung}

\newcommand{\Author}{René Schwaiger}
\newcommand{\Mail}{\href{mailto:sanssecours@me.com}{sanssecours@me.com}}

% ============
% = Settings =
% ============

% Boxes
\newtcbox{\codebox}{
  on line,
  arc = 2pt,
  colback = gray!10!white,
  boxsep = 2pt,
  left = 1pt,
  right = 1pt,
  top = 1pt,
  bottom = 1pt,
  boxrule = 0pt,
  bottomrule = 0pt,
  toprule = 0pt
}

\newtcolorbox{codeboxed}{
  on line,
  arc = 2pt,
  colback = gray!10!white,
  boxsep = 2pt,
  left = 1pt,
  right = 1pt,
  top = 1pt,
  bottom = 1pt,
  boxrule = 0pt,
  bottomrule = 0pt,
  toprule = 0pt
}

% Code
\usemintedstyle{rainbow_dash}
\BeforeBeginEnvironment{minted}{\begin{codeboxed}}
\AfterEndEnvironment{minted}{\end{codeboxed}}

% Colors
\definecolor{Aqua}{rgb}{0, 0.56, 1}

% Document Properties
\hypersetup{pdftitle=\Title, pdfsubject=\Subject, pdfauthor=\Author,
            pdfkeywords=\KeyWords}

\title{\texorpdfstring{\vspace{2cm}\\
       \hspace{-3mm}\textbf{\textcolor{black}{Y}
       \hspace{-3.6mm}\textcolor{red}{A}\\[-10pt]
       \hspace{+3mm}\textcolor{black}{ML}}\vspace{-1cm}}{YAML}}
\date{\textit{\Subject}}

% Fonts
\setmainfont{Avenir}
\setsansfont{Ubuntu}
\setmonofont{Ubuntu Mono}
\newfontfamily\Helvetica{Helvetica}

% Language
\setmainlanguage{english}
\setotherlanguage{german}

% Theme
\usetheme[headheight = 10pt, footheight = 10pt]{boxes}
\addheadbox{header}{~~ \Title \hfill
                    \insertframenumber{} / \inserttotalframenumber ~~}
\addfootbox{footer}{~~ \insertsection ~ \insertsubsection ~ \insertsubsubsection}
\beamertemplatenavigationsymbolsempty
\setbeamertemplate{itemize item}[circle]

\setbeamercolor{alerted text}{fg = orange}
\setbeamercolor{frametitle}{fg = Aqua}
\setbeamercolor{block title}{fg = black}
\setbeamercolor{itemize item}{fg = black}
\setbeamercolor{structure}{fg = black}
\setbeamercolor{title}{fg = Aqua}
\setbeamercolor{titlelike}{fg = Aqua}
\setbeamercolor{left footer}{fg = white, bg = black}
\setbeamercolor{right footer}{fg = white, bg = black}
\setbeamercolor{footer}{fg = white, bg = black}
\setbeamercolor{header}{fg = white, bg = black}
\setbeamercovered{transparent = 20}

\setbeamerfont{title}{series=\Helvetica, size={\fontsize{48}{48}}}

\AtBeginSection{\frame{\sectionpage}}

% ==========
% = Macros =
% ==========

\newcommand{\code}[1]{
  \codebox{\mintinline{yaml}|#1|}
}

% ============
% = Document =
% ============

\begin{document}

{
  \usebackgroundtemplate{
    \tikz\node[opacity = 0.1]{
      \parbox[c][\paperheight][c]{\paperwidth}{
        \centering
        \includegraphics[width = 0.85\textwidth]{Figures/Logo}
      }
    };
  }
  \begin{frame}[plain]
      \titlepage
  \end{frame}
}

\begin{frame}{Inhalt}
    \tableofcontents
\end{frame}

\section{🐪 Allgemeines}

\begin{frame}[fragile]{🐪 Allgemeines}
  \begin{minted}[autogobble]{yaml}
    - Abkürzung:
       - Früher: Yet Another Markup Language
       - Heute: YAML Ain’t Markup Language
    - Aktuelle Version: YAML 1.2 (2009)
    - Superset von JSON # seit YAML 1.2
    - 3 verschiedene Datentypen:
      - Skalar:
        - "Hello World"
        - '👋 🌍'
        - 123
      - Sequenz: [Text, 'Text', "Text", 123.5]
      - Map: {🔑: Value, see-no-evil monkey: 🙈}
    - Hauptziel: “YAML is easily readable by humans.”
    # Kompletter Text auf dieser Folie ist valides YAML
  \end{minted}
\end{frame}

\subsection{😱 Flow Style}

\begin{frame}[fragile]{😱 Flow Style}
  \begin{block}{Dokument in reinem Flow-Style}~\\
    \begin{minted}[autogobble]{yaml}
      [{'Abkürzung':
         [{'Früher': 'Yet Another Markup Language'},
          {'Heute': 'YAML Ain’t Markup Language'}]},
       {'Aktuelle Version': 'YAML 1.2 (2009)'},
       'Superset von JSON',
       {'3 verschiedene Datentypen':
          [{'Skalar': ['Hello World', '👋 🌍', 123]},
           {'Sequenz': ['Text', 'Text', 'Text', 123.5]},
           {'Map': {'🔑': 'Value',
                    'see-no-evil monkey': '🙈'}}]},
       {'Hauptziel': '“YAML is easily readable by humans.”'}]
      # Kompletter Text auf dieser Folie ist valides YAML
    \end{minted}
  \end{block}
\end{frame}

\section{🐣 Einfache Datentypen}

\subsection{📏  Skalar}

\begin{frame}[fragile]{📏 Skalar}
  \begin{itemize}
    \item Einfachster Typ
    \item Speichert Zahlen, Zeichenketten
    \item Beispiele
    \begin{itemize}
      \item Strings
      \begin{itemize}
        \item \codebox{\mintinline{yaml}|Plain String|}
        \item \codebox{\mintinline{yaml}|'Single Quoted String'|}
        \item \codebox{\mintinline{yaml}|"Double Quoted String"|}
        \item \codebox{\mintinline{yaml}|'1337'|}
        \item \codebox{\mintinline{yaml}|!!str 1337|} (Explizite Typspezifikation)
      \end{itemize}
      \item Ganzzahlen
      \begin{itemize}
        \item \codebox{\mintinline{yaml}|1337|}
        \item \codebox{\mintinline{yaml}|!!int 1337|} (Explizite Typspezifikation)
      \end{itemize}
      \item Fließkommazahlen
      \begin{itemize}
        \item \codebox{\mintinline{yaml}|0.1337|}
        \item \codebox{\mintinline{yaml}{!!float 0.1337}} (Explizite Typspezifikation)
      \end{itemize}
    \end{itemize}
  \end{itemize}
\end{frame}

\subsection{📋 Sequenz}

\begin{frame}[fragile]{📋 Sequenz}
  \begin{block}{Folge von Werten \textcolor{gray}{(Array, Feld, Liste)}}~\\
    \begin{minted}[autogobble]{yaml}
      # Flow-Style
      [🍎, 🍊, [Sugar, Eggs, Chocolate]]
      # Block-Style
      - 🍎
      - 🍊
      - - Sugar
        - Eggs
        - Chocolate
      # Block & Flow-Style
      - 🍎
      - 🍊
      - [Sugar, Eggs, Chocolate]
    \end{minted}
  \end{block}
\end{frame}

\subsection{🗺 Map}

\begin{frame}[fragile]{🗺 Map}
  \begin{block}{Ungeordnete Paare von Werten}
    \textcolor{gray}{Assoziatives (Array|Feld), Dictionary, Hash(map|table)?}\\\\
    \begin{minted}[autogobble]{yaml}
      # Flow-Style
      { Austria: Vienna,
        South Africa: {Executive: Pretoria,
          Judicial: Bloemfontein, Legislative: Cape Town}}
      # Block-Style
      Austria: Vienna
      South Africa:
        Executive:   Pretoria
        Judicial:    Bloemfontein
        Legislative: Cape Town
      # Complex Key (`?` optional für simple Keys)
      ? [Pretoria, Bloemfontein, Cape Town]
      : South Africa
    \end{minted}
  \end{block}
\end{frame}

\section{✨ Features}
\subsection{📝 Eingabe von Skalaren}

\subsubsection{🌊 Flow Style}

\begin{frame}[fragile]{🌊 Flow Style}
  \begin{enumerate}
    \item \code{Plain}
    \begin{itemize}
      \item Gute Lesbarkeit
      \item Kombinationen \codebox{\texttt{\#~~}} , \code{: ⁣} nicht möglich
      \item Indikatoren (\code{:}, \code{[}, …) nur in Spezialfällen als erstes Zeichen erlaubt
    \end{itemize}
    \item \code{'Single Quoted'}
    \begin{itemize}
      \item Ähnlich gut lesbar wie Plain Style
      \item Bis auf \code{'} keine Sonderzeichen
    \end{itemize}
    \item \code{"Double Quoted"}
    \begin{itemize}
      \item Unterstützung für Escape Sequences:\\ \code{\n}, \codebox{\texttt{{\textbackslash}x0a}}, \codebox{\texttt{{\textbackslash}u000a}}, \codebox{\texttt{{\textbackslash}U0000000a}}, …
      \item alle Strings darstellbar
    \end{itemize}
  \end{enumerate}
\end{frame}

\subsubsection{🗒 Block Style}

\begin{frame}[fragile, allowframebreaks]{🗒 Block Style}
  \begin{enumerate}

    \item
      \begin{minted}[autogobble]{yaml}
        |
          Literal
          Style
        # "Literal\nStyle"
      \end{minted}
      \begin{itemize}
        \item Newlines (\code{\n}) werden direkt übernommen
      \end{itemize}

    \item
      \begin{minted}[autogobble]{yaml}
        >
          Folded
          Style
        # "Folded Style"
      \end{minted}
      \begin{itemize}
        \item Erstes Newline-Zeichen wird durch Space ersetzt, falls es keine führenden Leerzeichen in der nächsten nicht-leeren Zeile gibt
    \end{itemize}

  \end{enumerate}

  \newpage
  \begin{tikzpicture}[overlay, remember picture]
  \node[anchor=center] at (current page.center) {
	\begin{columns}

    \column{.50\textwidth}
    \begin{minted}[autogobble, showspaces, spacecolor = lightgray, space=·]{yaml}
      >
        One
        Two

        List:
         * Item 1
         * Item 2
      # "One Two\nList:\n\ \
      #  * Item 1\n * Item 2\n"
    \end{minted}

    \column{.50\textwidth}
    \begin{minted}[autogobble, showspaces, spacecolor = lightgray, space=·]{yaml}
      |
        One
        Two

        List:
         * Item 1
         * Item 2
      # "One\nTwo\n\nList:\n\ \
      #  * Item 1\n * Item 2\n"
    \end{minted}
  \end{columns}};
  \end{tikzpicture}

  \newpage
  \begin{block}{Block Scalar Header}
    \begin{enumerate}
      \item Indentation Header \codebox{\texttt{|4}}, \code{>2}, \code{>5}, …
      \begin{itemize}
        \item Spezifiziert Einrückung des Blocks
        \item Kein Indentation Header $⇔$ Automatische Detektion
        \item Erste Zeilen enthalten zusätzliche Leerzeichen $⇔$ \\
              \alert{Fehler bei Auto-Detektion}
      \end{itemize}
      \item Chomping Header \codebox{\texttt{|-}}, \codebox{\texttt{|+}}, \code{>-}, \code{>+}
      \begin{itemize}
        \item Spezifiziert Bedeutung der leeren Zeilen \alert{am Ende}
        \begin{enumerate}
          \item Strip \code{-}: Leere Zeilen werden verworfen
          \item Keep \code{+}: Leere Zeilen werden gespeichert
          \item Clip (Kein Indikator): Leere Zeilen werden durch \code{\n} ersetzt
        \end{enumerate}
      \end{itemize}
    \end{enumerate}
    Beliebige Kombinationen \codebox{\texttt{|-8}}, \code{>4+}, … möglich
  \end{block}

\end{frame}

\subsection{🗃 Stream}

\begin{frame}[fragile]{🗃 Stream}
  \begin{itemize}
    \item YAML Stream: Mehrere Dokumente hintereinander
    \item Document-End-Marker \code{...} trennt einzelne Dokumente
    \item Optionaler Directives-End-Marker \code{---} trennt “Directives” von eigentlichem Inhalt
  \end{itemize}
  \begin{minted}[autogobble]{yaml}
    %YAML 1.2 # Directive
    ---
    "Hello First Document"
    ...
    Document without Directives
    ...
    %TAG ! example.uri
    ---
    'Third Document'
  \end{minted}
\end{frame}

\subsection{📢 Directives}

\begin{frame}[fragile]{📢 Directives}
  Anweisungen vor dem eigentlichen Dokument spezifizieren:
  \begin{enumerate}
    \item YAML Version \codebox{\texttt{\%YAML}}
    \item Prefix für Tag-Shorthands \codebox{\texttt{\%TAG}}
    \begin{itemize}
      \item Primary und Secondary Tag-Shorthand sind vordefiniert:
      \begin{minted}[autogobble]{yaml}
        %TAG !  !
        %TAG !! tag:yaml.org,2002:
      \end{minted}
    \end{itemize}
  \end{enumerate}
  \begin{minted}[autogobble]{yaml}
    # Directives
    %YAML 1.2                       # YAML Version
    %TAG ! tag:yaml.org,2002:       # Primary Tag Handle
    %TAG !! tag:yaml.org,2002:      # Secondary Tag Handle
    %TAG !named! tag:yaml.org,2002: # Named Tag Handle
    --- # Directives End Marker
    [!!str 6, !str 6, !named!str 6, ! 6,
     '6', !<tag:yaml.org,2002:str> 6] # 6 times "6"
  \end{minted}
\end{frame}

\subsection{🐠🐙🐶 Typen}

\subsubsection{🏷 Tags}

\begin{frame}[allowframebreaks, fragile]{🏷 Tags}
  \begin{itemize}
    \item Optionaler Tag spezifiziert Datentyp
    \item Ohne Tag $⇒$ automatische Erkennung von Typ
    \item 3 verschiedene Arten von Tags
    \begin{enumerate}
      \item Verbatim \code{!<!local>}, \code{!<global>}
      \item Shorthand \code{!suffix}, \code{!!suffix}, \code{!named!suffix}
      \item Non-Specific \code{!}
    \end{enumerate}
    \item Verbatim Tags
    \begin{description}
      \item [Local Tag] Unterschiedliche Bedeutung je nach Anwendung (Bsp: \code{!<!ruby/object:Set>})
      \item [Global Tag] Bedeutung eindeutig\\
            (Beispiel: \code{!<tag:yaml.org,2002:int>})
    \end{description}
    \item Non Specific Tag \code{!} (nur bei Non-Plain Skalar)
    \item Shorthand-Tags und Non-Specific Tags werden zu Verbatim Tag erweitert
  \end{itemize}

  \newpage

  \begin{block}{Umwandlung von Shorthand/Non-Specific $⇒$ Verbatim}~\\
    \begin{minted}[autogobble]{yaml}
      # %TAG  shorthand     definition
      %TAG     !name!    tag:yaml.org,2002:
      ---
      # shorthand|suffix ➡️ !<definition|suffix>
      !suffix      value # !<!suffix>                  value
      !!suffix     value # !<tag:yaml.org,2002:suffix> value
      !name!suffix value # !<tag:yaml.org,2002:suffix> value
      !            value # !<tag:yaml.org,2002:str>    value
      ...
      %TAG !name! lego.block/ # New prefix for shorthand
      ---
      !name!suffix value # !<lego.block/suffix> value
    \end{minted}
  \end{block}

\end{frame}

\subsubsection{🍳 Schemas}

\begin{frame}[allowframebreaks, fragile]{🍳 Schemas}
  \begin{itemize}
    \item 3 Vordefinierte Schema (Kategorien) für Typen (Tags)
    \begin{enumerate}
      \item Failsafe Schema
      \begin{description}
        \item [String] \code{!!str Text}
        \item [Sequence] \code{!!seq [Item1, Item2]}
        \item [Mapping] \code{!!map {Key: Value}}
      \end{description}
      \item JSON Schema (Failsafe Schema + JSON Typen)
      \begin{description}
        \item [Null] \code{!!null null: value of null key}
        \item [Boolean] \code{!!bool false}
        \item [Integer] \code{!!int -42}
        \item [Float] \code{!!float .nan}
      \end{description}
      \item Core Schema (JSON Schema + Erweiterte Daten-Eingabe)
      \begin{itemize}
        \item Octal- \code{0o123} und Hexadezimal-Format \code{0xfefe}
        \item Mehr Schreibweisen für spezielle Werte \\
              z.B. \code{null}, \code{Null}, \code{~}, \code{NULL}
      \end{itemize}
    \end{enumerate}
  \end{itemize}

  \newpage
  \begin{itemize}
    \item Zusätzliche Typen (\textcolor{orange}{\url{http://yaml.org/type}})
    \begin{description}[labelwidth = 1cm]
      \item [Ordered Map] \code{!!omap [first: 1, second: 2]}
      \item [Set] \code{!!set {No, Order}}
      \item [Binary] \code{!!binary "YmFzZTY0Cg=="}
      \item [Time] \code{!!timestamp 1337-12-24}
      \item […]
    \end{description}
    \item Eigene Datentypen z.B. mittels Local Tag~\\[0.5cm]
    \begin{minted}[autogobble]{yaml}
      # print(Set.new([false, 1, '2']).to_yaml)
      --- !ruby/object:Set
      hash:
        false: true
        1: true
        '2': true
    \end{minted}
  \end{itemize}
\end{frame}

\subsection{⚓️ Anchors \& Aliases}

\begin{frame}[fragile]{⚓️ Anchors \& Aliases}
  \begin{itemize}
    \item Dienen dazu selbes Datum öfters wiederzuverwenden
    \item Anchor \code{&name} verweist auf Datum
    \item Referenz \code{*name} fügt Datum an neuer Stelle ein~\\[0.5cm]
    \begin{minted}[autogobble]{yaml}
      green: &green
        🌳🌱🌿🌱🌿🌱🌿🌳
      flowers: &flowers
        🌳🌸🌼🌻🌺🌼🌸🌳
      garden:              # print(data['garden']):
        - *green           # 🌳🌱🌿🌱🌿🌱🌿🌳
        - *flowers         # 🌳🌸🌼🌻🌺🌼🌸🌳
        - 🌳🌱🐞🌱🌱🐛🌱🌳 # 🌳🌱🐞🌱🌱🐛🌱🌳
        - *flowers         # 🌳🌸🌼🌻🌺🌼🌸🌳
        - *green           # 🌳🌱🌿🌱🌿🌱🌿🌳
    \end{minted}
  \end{itemize}
\end{frame}

\end{document}
