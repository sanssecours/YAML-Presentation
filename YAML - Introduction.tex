%!TEX TS-program = xelatex

\documentclass{beamer}

% ============
% = Packages =
% ============

\usepackage{minted}
\usepackage{polyglossia}
\usepackage{tcolorbox}
\usepackage{tikz}
\usepackage{xltxtra}
\usepackage{xelatexemoji}

% ==============
% = Attributes =
% ==============

\newcommand{\Title}{YAML}
\newcommand{\TitleDescription}{Einführung}
\newcommand{\Subject}{Einführende Präsentation zur\texorpdfstring{\\}{}
                      Daten-Serialisierungssprache YAML}
\newcommand{\KeyWords}{YAML, Serialisierung}
\newcommand{\LeftFooter}{\Title~—~\TitleDescription}

\newcommand{\Author}{René Schwaiger}
\newcommand{\Mail}{\href{mailto:sanssecours@me.com}{sanssecours@me.com}}

% ============
% = Settings =
% ============

\newtcbox{\codebox}{
  on line,
  arc = 0pt,
  outer arc = 0pt,
  colback = gray!10!white,
  colframe = gray!50!black,
  boxsep = 0pt,
  left = 1pt,
  right = 1pt,
  top = 2pt,
  bottom = 2pt,
  boxrule = 0pt,
  bottomrule = 1pt,
  toprule = 1pt
}

% Code
\usemintedstyle{rainbow_dash}

% Colors
\definecolor{Aqua}{rgb}{0, 0.56, 1}

% Document Properties
\hypersetup{pdftitle=\Title, pdfsubject=\Subject, pdfauthor=\Author,
            pdfkeywords=\KeyWords}

\title{\texorpdfstring{\vspace{2cm}\\
       \hspace{-3mm}\textbf{\textcolor{black}{Y}
       \hspace{-3.6mm}\textcolor{red}{A}\\[-10pt]
       \hspace{+3mm}\textcolor{black}{ML}}}{YAML}}
\date{\textit{\Subject}}

% Fonts
\setmainfont{Avenir}
\setsansfont{Ubuntu}
\setmonofont{Ubuntu Mono}
\newfontfamily\Helvetica{Helvetica}

% Language
\setmainlanguage{english}
\setotherlanguage{german}

% Theme
\usetheme[headheight = 10pt, footheight = 10pt]{boxes}
\addheadbox{header}{~~ \Title ~~~~~~
                    \insertframenumber{} / \inserttotalframenumber}
\addfootbox{footer}{~~ \insertsection ~ \insertsubsection}
\beamertemplatenavigationsymbolsempty
\setbeamertemplate{itemize item}[circle]

\setbeamercolor{alerted text}{fg = orange}
\setbeamercolor{frametitle}{fg = Aqua}
\setbeamercolor{block title}{fg = black}
\setbeamercolor{itemize item}{fg = black}
\setbeamercolor{structure}{fg = black}
\setbeamercolor{title}{fg = Aqua}
\setbeamercolor{titlelike}{fg = Aqua}
\setbeamercolor{left footer}{fg = white, bg = black}
\setbeamercolor{right footer}{fg = white, bg = black}
\setbeamercolor{footer}{fg = white, bg = black}
\setbeamercolor{header}{fg = white, bg = black}
\setbeamercovered{transparent = 20}

\setbeamerfont{title}{series=\Helvetica, size={\fontsize{48}{48}}}

% ==========
% = Macros =
% ==========

\newcommand{\code}[1]{
  \codebox{\mintinline{yaml}|#1|}
}

% ============
% = Document =
% ============

\begin{document}

{
  \usebackgroundtemplate{
    \tikz\node[opacity = 0.1]{
      \parbox[c][\paperheight][c]{\paperwidth}{
        \centering
        \includegraphics[width = 0.85\textwidth]{Figures/Logo}
      }
    };
  }
  \begin{frame}[plain]
      \titlepage
  \end{frame}
}

\begin{frame}{Inhalt}
    \tableofcontents
\end{frame}

\section{🐪 Allgemeines}

\begin{frame}[fragile]{🐪 Allgemeines}
  \begin{minted}[autogobble]{yaml}
    - Abkürzung:
       - Früher: Yet Another Markup Language
       - Heute: YAML Ain’t Markup Language
    - Aktuelle Version: YAML 1.2 (2009)
    - Superset von JSON # seit YAML 1.2
    - 3 verschiedene Datentypen:
      - Skalar:
        - "Hello World"
        - '👋 🌍'
        - 123
      - Sequenz: [Text, 'Text', "Text", 123.5]
      - Map: {🔑: Value, see-no-evil monkey: 🙈}
    - Hauptziel: “YAML is easily readable by humans.”
    # Kompletter Text auf dieser Folie ist valides YAML
  \end{minted}
\end{frame}

\subsection{😱 Flow Style}

\begin{frame}[fragile]{😱 Flow Style}
  \begin{block}{Dokument der vorhergehenden Folie in reinem Flow-Style}
    \begin{minted}[autogobble]{yaml}
    [{'Abkürzung':
       [{'Früher': 'Yet Another Markup Language'},
        {'Heute': 'YAML Ain’t Markup Language'}]},
     {'Aktuelle Version': 'YAML 1.2 (2009)'},
     'Superset von JSON',
     {'3 verschiedene Datentypen':
        [{'Skalar': ['Hello World', '👋 🌍', 123]},
         {'Sequenz': ['Text', 'Text', 'Text', 123.5]},
         {'Map': {'🔑': 'Value', 'see-no-evil monkey': '🙈'}}]},
     {'Hauptziel': '“YAML is easily readable by humans.”'}]
    # Kompletter Text auf dieser Folie ist valides YAML
    \end{minted}
  \end{block}
\end{frame}

\section{⚪️ Einfache Datentypen}

\subsection{📏  Skalar}

\begin{frame}[fragile]{📏 Skalar}
  \begin{itemize}
    \item Einfachster Typ
    \item Speichert Zahlen, Zeichenketten
    \item Beispiele
    \begin{itemize}
      \item Strings
      \begin{itemize}
        \item \mintinline{yaml}|Plain String|
        \item \mintinline{yaml}|'Single Quoted String'|
        \item \mintinline{yaml}|"Double Quoted String"|
        \item \mintinline{yaml}|'1337'|
        \item \mintinline{yaml}|!!str 1337 # Explizite Typspezifikation|
      \end{itemize}
      \item Ganzzahlen
      \begin{itemize}
        \item \mintinline{yaml}|1337|
        \item \mintinline{yaml}|!!int 1337 # Explizite Typspezifikation|
      \end{itemize}
      \item Fließkommazahlen
      \begin{itemize}
        \item \mintinline{yaml}|0.1337|
        \item \mintinline{yaml}|!!float 0.1337 # Explizite Typspezifikation|
      \end{itemize}
    \end{itemize}
  \end{itemize}
\end{frame}

\subsection{📋 Sequenz}

\begin{frame}[fragile]{📋 Sequenz}
  \begin{block}{Folge von Werten \textcolor{gray}{(Array, Feld, Liste)}}
    \begin{minted}[autogobble]{yaml}
      # Flow-Style
      [🍎, 🍊, [Sugar, Eggs, Chocolate]]
      # Block-Style
      - 🍎
      - 🍊
      - - Sugar
        - Eggs
        - Chocolate
      # Block & Flow-Style
      - 🍎
      - 🍊
      - [Sugar, Eggs, Chocolate]
    \end{minted}
  \end{block}
\end{frame}

\subsection{🗺 Map}

\begin{frame}[fragile]{🗺 Map}
  \begin{block}{Ungeordnete Paare von Werten}
    \textcolor{gray}{Assoziatives (Array|Feld), Dictionary, Hash(map|table)?}\\
    \begin{minted}[autogobble]{yaml}
      # Flow-Style
      { Austria: Vienna,
        South Africa: {Executive: Pretoria,
          Judicial: Bloemfontein, Legislative: Cape Town}}
      # Block-Style
      Austria: Vienna
      South Africa:
        Executive:   Pretoria
        Judicial:    Bloemfontein
        Legislative: Cape Town
    \end{minted}
  \end{block}
\end{frame}

% - Datentypen
% - Directives
% - Streams
\section{✨ Features}
\subsection{✒️ Skalare}

\subsubsection{🌊 Flow Style}

\begin{frame}[fragile]{🌊 Flow Style}
  \begin{enumerate}
    \item \code{Plain}
    \begin{itemize}
      \item Gute Lesbarkeit
      \item Kombinationen \codebox{\#~~~} , \code{: ⁣} nicht möglich
      \item Indikatoren (\code{:}, \code{[}, …) nur in Spezialfällen als erstes Zeichen erlaubt
    \end{itemize}
    \item \code{'Single Quoted'}
    \begin{itemize}
      \item Ähnlich gut lesbar wie Plain Style
      \item Bis auf \code{'} keine Sonderzeichen
    \end{itemize}
    \item \code{"Double Quoted"}
    \begin{itemize}
      \item Unterstützung für Escape Sequences:\\ \code{\n}, \codebox{{\textbackslash}x0a}, \codebox{{\textbackslash}u000a}, \codebox{{\textbackslash}U0000000a}, …
      \item alle Strings darstellbar
    \end{itemize}
  \end{enumerate}
\end{frame}

\subsubsection{🗒 Block Style}

\begin{frame}[fragile, allowframebreaks]{🗒 Block Style}
  \begin{enumerate}
    \item
    \begin{minted}[autogobble, bgcolor = gray!10!white]{yaml}
      |
        Literal
    \end{minted}
    \begin{itemize}
      \item Newlines (\code{\n}) werden direkt übernommen
    \end{itemize}
    \item
    \begin{minted}[autogobble, bgcolor = gray!10!white]{yaml}
      >
        Folded
    \end{minted}
    \begin{itemize}
      \item Erstes Newline-Zeichen wird durch Space ersetzt, falls es keine führenden Leerzeichen in der nächsten nicht-leeren Zeile gibt
    \end{itemize}
  \end{enumerate}
\end{frame}

\end{document}
